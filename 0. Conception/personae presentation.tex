\documentclass{article}
\usepackage[utf8]{inputenc}
\usepackage{graphicx}

\title{AI Assistant - Personas}
\author{Loic Broihier, Paul Jaulin, Baptiste Guillemin}
\date{Juin 2024}

\begin{document}

\maketitle

\section*{Personas}

\subsection*{Persona 1: Alice, Étudiante en Master}
\begin{itemize}
    \item \textbf{Nom}: Alice Dupont
    \item \textbf{Âge}: 23 ans
    \item \textbf{Profession}: Étudiante en Master de Biologie
    \item \textbf{Niveau de compétence technologique}: Élevé
    \item \textbf{Besoins}: Alice a besoin d'un outil pour résumer rapidement des articles scientifiques et extraire les informations pertinentes pour ses recherches et ses devoirs.
    \item \textbf{Motivations}: Gagner du temps, améliorer la qualité de ses résumés, se préparer efficacement pour ses examens et ses projets de recherche.
    \item \textbf{Contexte d'utilisation}: Utilise l'application sur son ordinateur portable à la bibliothèque ou à la maison, souvent sous pression de temps avant des échéances importantes.
    \item \textbf{Objectifs}: Obtenir des résumés précis et des réponses rapides à ses questions spécifiques concernant des articles scientifiques.
\end{itemize}

\subsection*{Persona 2: Jean, Avocat en Cabinet Privé}
\begin{itemize}
    \item \textbf{Nom}: Jean Martin
    \item \textbf{Âge}: 45 ans
    \item \textbf{Profession}: Avocat
    \item \textbf{Niveau de compétence technologique}: Moyen
    \item \textbf{Besoins}: Jean a besoin d'un outil pour extraire et résumer les informations clés des documents juridiques volumineux, et pour obtenir des réponses précises à des questions juridiques complexes.
    \item \textbf{Motivations}: Réduire le temps passé à lire et analyser les documents, améliorer l'efficacité du travail, fournir des conseils juridiques précis et rapides à ses clients.
    \item \textbf{Contexte d'utilisation}: Utilise l'application principalement sur son bureau au travail, parfois sur une tablette lors de déplacements ou de réunions avec des clients.
    \item \textbf{Objectifs}: Optimiser son temps de travail, améliorer la précision de ses analyses juridiques, offrir un meilleur service à ses clients.
\end{itemize}

\subsection*{Persona 3: Marie, Chef de Projet en Entreprise Technologique}
\begin{itemize}
    \item \textbf{Nom}: Marie Leblanc
    \item \textbf{Âge}: 35 ans
    \item \textbf{Profession}: Chef de Projet
    \item \textbf{Niveau de compétence technologique}: Très élevé
    \item \textbf{Besoins}: Marie a besoin d'un outil pour synthétiser des rapports techniques et des documents de projet, et pour poser des questions spécifiques concernant l'avancement des projets et les problèmes techniques.
    \item \textbf{Motivations}: Améliorer la communication et la coordination au sein de son équipe, accélérer la prise de décision, maintenir une vue d'ensemble précise des projets en cours.
    \item \textbf{Contexte d'utilisation}: Utilise l'application sur son ordinateur de bureau au travail, parfois sur son smartphone pendant ses déplacements ou lors de réunions.
    \item \textbf{Objectifs}: Faciliter la gestion de projet, améliorer l'efficacité de son équipe, assurer la réussite des projets dans les délais impartis.
\end{itemize}

\end{document}
